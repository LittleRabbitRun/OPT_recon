\chapter{Concepts}
\setheader{{\it CHAPTER \thechapter}}{}{}{\ctsimheadtitle}{}{{\it CHAPTER \thechapter}}
\ctsimfooter

\section{Conceptual Overview}\index{Conceptual overview}
The operation of \ctsim\ begins with the phantom object.  A
phantom object consists of geometric elements.  A scanner is
specified and the collection of x-ray data, or projections, is
simulated. This projection data can be reconstructed using various
user-controlled algorithms producing an image of the phantom
object. These reconstructions can be visually and statistically
compared to the original phantom object.

In order to use \ctsim\ effectively, some knowledge of how
\ctsim\ works and the approach taken is required. \ctsim\ deals with a
variety of object, but the two primary objects that we need to be
concerned with are the \helprefn{phantom}{conceptphantom} and the
\helprefn{scanner}{conceptscanner}.

\section{Phantoms}\label{conceptphantom}

\ctsim\ uses geometrical objects to describe the object being
scanned. A phantom is composed of one or more phantom elements.
These elements are simple geometric shapes, specifically,
rectangles, triangles, ellipses, sectors and segments. With these
elements, the standard phantoms used in the CT literature can be
constructed.  In fact, \ctsim\ provides a shortcut to load the
published phantoms of Herman\cite{HERMAN80} and
Shepp-Logan\cite{SHEPP74}. \ctsim\ also reads text files of
user-defined phantoms.

The types of phantom elements and their definitions are taken with
permission from G.T. Herman's publication\cite{HERMAN80}.

\subsection{Phantom File}\label{phantomfile}\index{Phantom!File syntax}
Each line in the text file describes an element of the
phantom.  Each line contains seven entries, in the following form:
\begin{verbatim}
element-type cx cy dx dy r a
\end{verbatim}
The first entry defines the type of the element, either
\texttt{rectangle}, \texttt{ellipse}, \texttt{triangle},
\texttt{sector}, or \texttt{segment}.

For all phantom elements, \texttt{r} is the rotation applied to the object in degrees
counterclockwise and \texttt{a} is the X-ray attenuation
coefficient of the object. Where objects overlap, the attenuations
of the overlapped objects are summed.

As opposed to the \texttt{r} and \texttt{a} fields, the \texttt{cx},
\texttt{cy}, \texttt{dx} and \texttt{dy} fields have different
meanings depending on the element type.



\subsection{Phantom Elements}\label{phantomelements}\index{Phantom!Elements}

\subsubsection{ellipse}
Ellipses use \texttt{dx} and \texttt{dy} to define the semi-major and
semi-minor axis lengths with the center of the ellipse at \texttt{(cx,cy)}.
Of note, the commonly used phantom described by
Shepp and Logan\cite{SHEPP74} uses only ellipses.

\subsubsection{rectangle}
Rectangles use \texttt{(cx,cy)} to define the position of
the center of the rectangle with respect to the origin.  \texttt{dx}
and \texttt{dy} are the half-width and half-height of the rectangle.

\subsubsection{triangle}
Triangles are drawn with the center of the base at \texttt{(cx,cy)}
and a base half-width of \texttt{dx} and a height of \texttt{dy}.
Rotations are then applied about the center of the base.

\subsubsection{segment}
Segments are complex. They are the portion of an circle between a
chord and the perimeter of the circle.  \texttt{dy} sets the
radius of the circle. Segments start with the center of the chord
located at \texttt{(0,0)} and the chord horizontal. The half-width
of the chord is set by \texttt{dx}.  The portion of an circle
lying below the chord is then added. The imaginary center of this
circle is located at \texttt{(0,-dy)}. The segment is then rotated
by \texttt{r} and then translated by \texttt{(cx,cy)}.

\subsubsection{sector}
Sectors are the like a ``pie slice'' from a circle. The radius of
the circle is set by \texttt{dy}. Sectors are defined similarly to
segments. In this case, though, a chord is not drawn.  Instead,
the lines are drawn from the origin of the circle \texttt{(0,-dy)}
to the points \texttt{(-dx,0)} and \texttt{(dx,0)}. The perimeter
of the circle is then drawn between those two points and lies
below the x-axis. The sector is then rotated and translated the
same as a segment.

\subsection{Phantom Size}\index{Phantom!Size}
The overall dimensions of the phantom are increased by 1\% above the
specified sizes to avoid clipping due to round-off errors from
sampling the polygons of the phantom elements.  So, if the phantom is
defined as a rectangle of size 0.1 by 0.1, the phantom size is
0.101 in each direction.

\section{Scanner}\label{conceptscanner}\index{Scanner!Concepts}%
Understanding the scanning geometry is the most complicated aspect of
using \ctsim. For real-world CT simulators, this is actually quite
simple. The geometry is fixed by the manufacturer during the
construction of the scanner and can not be changed. \ctsim,
being a very flexible simulator, gives tremendous options in
setting up the geometry for a scan.

\subsection{Dimensions}
The geometry for a scan starts with the size of
the phantom being scanned. This is because \ctsim\ allows for
statistical comparisons between the original phantom image and
it's reconstructions. Since CT scanners scan a circular area, the
first important variable is the diameter of the circle surround
the phantom, the \emph{phantom diameter}. Remember, as mentioned
above, the phantom dimensions are padded by 1\%.

The other important geometry variables for scanning phantoms are
the \emph{view diameter}, \emph{scan diameter}, \emph{focal
length}, and \emph{center-detector length}. These variables are input into \ctsim\ in terms of
ratios rather than absolute values.

\subsubsection{Phantom Diameter}\index{Phantom!Diameter}
The phantom diameter is automatically calculated by \ctsim\ from
the phantom definition. The maximum of the phantom length and
height is used to define the square that completely surrounds the
phantom. Let \latexonly{$p_l$}\latexignore{\emph{Pl}} be the width
and height of this square. The diameter of this boundary box,
\latexonly{$p_d$,}\latexignore{\emph{Pd},} is given by the
Pythagorean theorem and is
\latexignore{\\\centerline{\emph{Pl x sqrt(2)}}\\}
\latexonly{\begin{equation}p_d = p_l \sqrt{2}\end{equation}}
CT scanners collect projections around a
circle rather than a square. The diameter of this circle is
the diameter of the boundary square \latexonly{$p_d$.}
\latexignore{\emph{Pd}.}
\latexonly{These relationships are diagrammed in figure~\ref{phantomgeomfig}.}
\begin{figure}
\centerline{\image{8cm;0cm}{scangeometry.eps}}
\latexonly{\caption{\label{phantomgeomfig} Phantom Geometry}}
\end{figure}

\subsubsection{View Diameter}\index{View diameter}
The \emph{view diameter} is the area that is being processed
during scanning of phantoms as well as during rasterization of
phantoms. By default, the \emph{view diameter} is set equal
to the \emph{phantom diameter}. It may be useful, especially for
experimental reasons, to process an area larger (and maybe even
smaller) than the phantom. Thus, during rasterization or during
projections, \ctsim\ will ask for a \emph{view ratio},
\latexonly{$v_r$.}\latexignore{\rtfsp \emph{VR}.} The \emph{view
diameter} is then calculated as
\latexonly{\begin{equation}v_d = p_dv_r\end{equation}}
\latexignore{\\\centerline{\emph{Vd = Pd x VR}}\\}

By using a
\latexonly{$v_r$}\latexignore{\emph{VR}}
less than 1, \ctsim\ will allow
for a \emph{view diameter} less than
\emph{phantom diameter}.
This will lead to significant artifacts. Physically, this would
be impossible and is analogous to inserting an object into the CT
scanner that is larger than the scanner itself!

\subsubsection{Scan Diameter}\index{Scan diameter}
By default, the entire \emph{view diameter} is scanned. For
experimental purposes, it may be desirable to scan an area either
larger or smaller than the \emph{view diameter}. Thus, the concept
of \emph{scan ratio}, \latexonly{$s_r$,}\latexignore{\emph{SR},}
arises. The scan diameter,
\latexonly{$s_d$,}\latexignore{\emph{Sd},} is the diameter over
which x-rays are collected and is defined as
\latexonly{\begin{equation}s_d =v_d s_r\end{equation}}
\latexignore{\\\centerline{\emph{Sd = Vd x SR}}\\}
By default and
for all ordinary scanning, the \emph{scan ratio} is to \texttt{1}.
If the \emph{scan ratio} is less than \texttt{1}, you can expect
significant artifacts.

\subsubsection{Focal Length}\index{Focal length}
The \emph{focal length},
\latexonly{$f$,}\latexignore{\emph{F},}
is the distance of the X-ray source to the center of
the phantom. The focal length is set as a ratio,
\latexonly{$f_r$,}\latexignore{\emph{FR},}
of the view radius. Focal length is
calculated as
\latexonly{\begin{equation}f = (v_d / 2) f_r\end{equation}}
\latexignore{\\\centerline{\emph{F = (Vd / 2) x FR}}}

For parallel geometry scanning, the focal length doesn't matter.
However, for divergent geometry scanning (equilinear and equiangular),
the \emph{focal length ratio} should be set at \texttt{2} or more
to avoid artifacts. Moreover, a value of less than \texttt{1} is
physically impossible and it analagous to having the x-ray
source inside of the \emph{view diameter}.

\subsubsection{Center-Detector Length}\index{Center-Detector length}
The \emph{center-detector length},
\latexonly{$c$,}\latexignore{\emph{C},}
is the distance from the center of
the phantom to the center of the detector array. The center-detector length is set as a ratio,
\latexonly{$c_r$,}\latexignore{\emph{CR},}
of the view radius. The center-detector length is
calculated as
\latexonly{\begin{equation}f = (v_d / 2) c_r\end{equation}}
\latexignore{\\\centerline{\emph{F = (Vd / 2) x CR}}}

For parallel geometry scanning, the center-detector length doesn't matter.
A value of less than \texttt{1} is physically impossible and it analagous to
having the detector array inside of the \emph{view diameter}.


\subsection{Parallel Geometry}\label{geometryparallel}\index{Parallel geometry}\index{Scanner!Parallel}

The simplest geometry, parallel, was used in first generation
scanners. As mentioned above, the focal length is not used in this simple
geometry. The detector array is set to be the same size as the
\emph{scan diameter}.  For optimal scanning in this geometry, the
\emph{scan diameter} should be equal to the \emph{phantom
diameter}. This is accomplished by using the default values of
\texttt{1} for the \emph{view ratio} and the \emph{scan ratio}. If
values of less than \texttt{1} are used for these two variables,
significant distortions will occur.


\subsection{Divergent Geometries}\label{geometrydivergent}\index{Equilinear geometry}\index{Equiangular geometry}
\index{Scanner!Equilinear}\index{Scanner!Equiangular}
For both equilinear (second generation) and equiangular
(third, fourth, and fifth generation) geometries,
the x-ray beams diverge from a single source to a detector array.
In the equilinear mode, a single
source produces a fan beam which is read by a linear array of detectors.  If
the detectors occupy an arc of a circle, then the geometry is equiangular.
\latexonly{These configurations are shown in figure~\ref{divergentfig}.}
\begin{figure}
\centerline{\image{10cm;0cm}{divergent.eps}}
\latexonly{\caption{\label{divergentfig} Equilinear and equiangular geometries.}}
\end{figure}


\subsubsection{Fan Beam Angle}\index{Fan beam angle}
For these divergent beam geometries, the \emph{fan beam angle}
needs to be calculated. For real-world CT scanners, this is fixed
at the time of manufacture. \ctsim, however, calculates the
\emph{fan beam angle}, $\alpha$, from the \emph{scan diameter} and
the \emph{focal length} as
\latexignore{\centerline{\emph{alpha = 2 x asin (
(Sd / 2) / f)}}}
\latexonly{\begin{equation}\label{alphacalc}\alpha = 2 \sin^{-1}
((s_d / 2) / f)\end{equation}}
\latexonly{This is illustrated in figure~\ref{alphacalcfig}.}
\begin{figure}
\centerline{\image{10cm;0cm}{alphacalc.eps}}
\latexonly{\caption{\label{alphacalcfig} Calculation of $\alpha$}}
\end{figure}


Empiric testing with \ctsim\ shows that for very large \emph{fan beam angles},
greater than approximately
\latexonly{$120^\circ$,}\latexignore{120 degrees,}
there are significant artifacts. The primary way to manage the
\emph{fan beam angle} is by varying the \emph{focal length} since the
\emph{scan diameter} is usually fixed at the size of the phantom.

To illustrate, the \emph{scan diameter} can be defined as
\latexonly{\begin{equation}s_d = s_r v_r p_d\end{equation}}
\latexignore{\\\centerline{\emph{Sd = Sr x Vr x Pd}}\\}

Further, the \emph{focal length} can be defined as
\latexonly{\begin{equation} f = f_r (v_r p_d / 2)\end{equation}}
\latexignore{\\\centerline{\emph{F = FR x (VR x Pd)$$\\}}}

Substituting these equations into \latexignore{the above
equation,}\latexonly{equation~\ref{alphacalc},} We have,
\latexonly{
\begin{eqnarray}
\alpha &=& 2\,\sin^{-1} \frac{\displaystyle s_r v_r p_d / 2}{\displaystyle f_r v_r (p_d / 2)} \nonumber \\
&=& 2\,\sin^{-1} (s_r / f_r)
\end{eqnarray}
} \latexignore{\\\centerline{\emph{\alpha = 2 sin (Sr / Fr)}}\\}

Since in normal scanning $s_r$ = 1, $\alpha$ depends only upon the
\emph{focal length ratio} in normal scanning.

\subsubsection{Detector Array Size}
In general, you do not need to be concerned with the detector
array size -- it is automatically calculated by \ctsim. For the
particularly interested, this section explains how the detector
array size is calculated.

For parallel geometry, the detector length is simply the scan
diameter.

For divergent beam geometries, the size of the detector array also
depends upon the \emph{focal length}: increasing the \emph{focal
length} decreases the size of the detector array.

For equiangular geometry, the detectors are equally spaced around a arc
covering an angular distance of $\alpha$ as viewed from the source. When
viewed from the center of the scanning, the angular distance is
\latexonly{$$\pi + \alpha - 2 \, \cos^{-1} \Big( \frac{s_d / 2}{c} \Big)$$}
\latexignore{\\\emph{pi + \alpha - 2 x acos ((Sd / 2) / C))}\\}
The dotted circle
\latexonly{in figure~\ref{equiangularfig}}
indicates the positions of the detectors in this case.

\begin{figure}
\centerline{\image{10cm;0cm}{equiangular.eps}}
\latexonly{\caption{\label{equiangularfig}Equiangular geometry}}
\end{figure}

For equilinear geometry, the detectors are equally spaced along a straight
line. The detector length depends upon
\latexonly{$\alpha$}\latexignore{\emph{alpha}} and the \emph{focal
length}. This length,
\latexonly{$d_l$,}\latexignore{Dl,} is calculated as
\latexonly{\begin{equation} d_l = 2\,(f + c) \tan (\alpha / 2)\end{equation}}
\latexignore{\\\centerline{\emph{2 x (F + C) x tan(\alpha/2)}}}
\latexonly{This geometry is shown in figure~\ref{equilinearfig}.}
\begin{figure}
\centerline{\image{10cm;0cm}{equilinear.eps}}
\latexonly{\caption{\label{equilinearfig} Equilinear geometry}}
\end{figure}


\section{Reconstruction}\label{conceptreconstruction}\index{Reconstruction overview}%

\subsection{Direct Inverse Fourier}
This method is not currently implemented in \ctsim; however, it is
planned for a future release. This method does not give results as
accurate as filtered backprojection. This is due primarily
to interpolation occurring in the frequency domain rather than the
spatial domain.

\subsection{Filtered Backprojection}\index{Filtered backprojection}\index{Symmetric multiprocessing}\index{SMP}
The technique is comprised of two sequential steps:
filtering projections followed by backprojecting the filtered projections. Though
these two steps are sequential, each view position can be processed independently.

\subsubsection{Parallel Computer Processing}\index{Parallel processing}
Since each view can be processed independently, filtered backprojection is amendable to
parallel processing. Indeed, this has been used in commercial scanners to speed reconstruction.
This parallelism is exploited both in the \ctsim\ graphical shell and
in the \helpref{LAM}{ctsimtextlam} version of \ctsimtext. \ctsim\ can distribute it's workload
amongst multiple processors working in parallel.

The graphical shell will automatically take advantage of multiple CPU's when
running on a \emph{Symmetric Multiprocessing}
computer. Dual-CPU computers are commonly available which provide a near doubling
in reconstruction speeds. \ctsim, though, has no limits on the number of CPU's
that can be used in parallel. The \emph{LAM} version
of \ctsimtext\ is designed to work in a cluster of computers.
This has been testing with a cluster of 16 computers in a
\urlref{Beowulf-class}{http://www.beowulf.org} cluster with excellent
results.

\subsubsection{Filter projections}
The first step in filtered backprojection reconstructions is the filtering
of each projection. The projections for a each view have their frequency data multipled by
a filter of $|w|$. \ctsim\ permits four different ways to accomplish this
filtering.

Two of the methods use convolution of the projection data with the
inverse Fourier transform of $|w|$. The other two methods perform an Fourier
transform of the projection data and multiply that by the $|w|$ filter and
then perform an inverse fourier transform.

Though multiplying by $|w|$ gives the sharpest reconstructions, in
practice, superior results are obtained by reducing the higher
frequencies. This is performed by mutiplying the $|w|$ filter by
another filter that attenuates the higher frequencies. \ctsim\ has
multiple filters for this purpose.

\subsubsection{Backprojection of filtered projections}
Backprojection is the process of ``smearing'' the filtered
projections over the reconstructing image. Various levels of
interpolation can be specified.

\section{Image Comparison}\label{conceptimagecompare}\index{Image!Comparison}
Images can be compared statistically. Three measurements can be calculated
by \ctsim. They are taken from the standard measurements used by
Herman\cite{HERMAN80}. They are:

\begin{itemize}\itemsep=0pt
\item[]\textbf{$d$}\quad The normalized root mean squared distance measure.
\item[]\textbf{$r$}\quad The normalized mean absolute distance measure.
\item[]\textbf{$e$}\quad The worst case distance measure over a \latexonly{$2\times2$}\latexignore{\emph{2 x 2}} pixel area.
\end{itemize}

These measurements are defined in equations \ref{dequation} through \ref{bigrequation}.
In these equations, $p$ denotes the phantom image, $r$ denotes the reconstruction
image, and $\bar{p}$ denotes the average pixel value of $p$. Each of the images have a
size of $m \times n$. In equation \ref{eequation} $[n/2]$ and $[m/2]$ denote the largest
integers less than $n/2$ and $m/2$, respectively.

\latexignore{These formulas are shown in the print documentation of \ctsim.}
%
%Tex2RTF can not handle the any subscripts or superscripts for the inner summation unless
% have a space character before the \sum
\latexonly{\begin{equation}\label{dequation} d =\sqrt{\frac{\displaystyle \sum_{i=1}^{n}{ \sum_{j=1}^{m}{(p_{i,j} - r_{i,j})^2}}}{\displaystyle \sum_{i=1}^{n}{ \sum_{j=1}^{m}{(p_{i,j} - \bar{p})^2}}}}\end{equation}}
\latexonly{\begin{equation}\label{requation}r = \frac{ \displaystyle \sum_{i=1}^{n}{ \sum_{j=1}^{m}{|p_{i,j} - r_{i,j}|}}}{ \displaystyle \sum_{i=1}^{n}{ \sum_{j=1}^{m}{|p_{i,j}|}}}\end{equation}}
\latexonly{\begin{equation}\label{eequation}e = \max_{1 \le k \le [n/2] \atop 1 \le l \le [m/2]}(|P_{k,l} - R_{k,l}|)\end{equation}}
\latexonly{where}
\latexonly{\begin{equation}\label{bigpequation}P_{k,l} = \textstyle \frac{1}{4} (p_{2k,2l} + p_{2k+1,2l} + p_{2k,2l+l} + p_{2k+1,2l+1})\end{equation}}
\latexonly{\begin{equation}\label{bigrequation}R_{k,l} = \textstyle \frac{1}{4} (r_{2k,2l} + r_{2k+1,2l} + r_{2k,2l+1} + r_{2k+1,2l+1})\end{equation}}
\begin{comment}
\end{comment}
